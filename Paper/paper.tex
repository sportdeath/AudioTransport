\documentclass[12pt]{article}

\usepackage{amsthm}
\usepackage{amsmath}
\usepackage{amsfonts}
\usepackage{graphicx}
\usepackage{url}
\usepackage{algorithm}
\usepackage{algpseudocode}

\newcommand{\fft}{\text{FFT}}
\newcommand{\ifft}{\text{IFFT}}

\title{Audio Transport}
\date{\today}
\author{Trevor Henderson}

\begin{document}
\maketitle

\begin{abstract}
  yoyoyo.
\end{abstract}

\section{Introduction}

\section{Contributions}

I developed an algorithm that

techniques to
reconstruct an audio signal

I use a new technique to segment the spectrum into smaller pieces.
Phase reconstruction

I also implemented the algorithm in real-time to 

\section{Related Work}
Vocoders existed in the 70's.
vocoders take a monophonic source and spread it out over a range of pitches. requires harmonic similarity between the sources. Voices have lots of noise so it is easy - Daft punk. 

realtime pitch shifting has become very popular in software and hardware. since the early 2000's.
Traditional pitch shifters were granular.
Updates to algorithms use phase vocoders.

Real time pitch shifting is a key element to autotune, which charachterized the music of the mid 2000's and is still popular today.

Harmonizers sound similar to vocoders. Take a pitch and pitch shift it into new chords.

In 2008 Melodyne pitch shifts individual elements without distoring the sound.
They must be processed beforehand and then can be manipulated in realtime

The 2012 product can manipulate polyphonic audio in realtime.
http://www.zynaptiq.com/pitchmap/

Most of this work has been proprietary and out of research. 

So far as I can 

processing but no synthesis.
\url{http://www.maths.lu.se/fileadmin/maths/personal_staff/Andreas_Jakobsson/ElvanderAKJ16_icassp_final.pdf}

http://remi.flamary.com/biblio/flamary2016optimal.pdf

\section{Optimal Transport}

Discussion of optimal transport.

Distance. 

It also allows for you to interpolate between the two shapes along the trajectory.

In 1 dimension it is a solved problem.

For any dimensionality higher than 1, fast computation of the wasserstein2 distance is 

Optimal transport is however only defined for, nonnegative masses.
Audio has phase.
Initial attempts to tackle the problem of phase involved formulating the transport problem differently.
But you can't hear the distance between two sinusoids that are played at the same pitch but at different phases
But you can hear the ``distance'' between two pitches.

\section{Phase Vocoders}
As noted by, 
audio quality is .

Vertical and horizontal incoherence.


\section{Algorithm Overview}

\subsection{STFT}

Like almost all spectral algorithms, this one begins with a short-time Fourier transform (STFT) and ends with it's inverse (ISTFT).
At the cost of reduced temporal resolution, the STFT gives us access to the spectral components of the signal in time.

The STFT performs a discrete time Fourier transforms on windows of size $M$ of the original signal $x(n)$.
These windows are separated by a hop size of $R$. The $\tau$th frame of the STFT is
\begin{align}
  X(\omega, \tau) = 
  \text{DTFT}(\tilde{x}(n, \tau)w(n))
\end{align}
where
\begin{align}
  \tilde{x}(n,\tau) = x(n + \tau R)
\end{align}
is a buffer of the original signal, shifted $R$ samples from the previous buffer and $w(n)$ is the synthesis window.
$w(n)$ is defined to be nonzero for $M$ samples, 
$n\in\left[-\frac{M - 1}{2}, \frac{M - 1}{2}\right]$, so only the $M$ most recent samples of $x(n)$ need to be kept in memory.

After some computation, 
we produce frames $Y(\omega, \tau)$ 
which we want to use to synthesize the signal $y(n)$.
We perform this ISTFT using the weighted overlap add method.
As the name suggests, this method works by
overlapping adjacent frames of the IDTFT, 
weighting them by an analysis window $f(n)$ 
and then adding them together.
The weighting helps to suppress audible discontinuities at frame boundaries
which prevents artifacts in highly nonlinear filters such as the one this paper describes.
\begin{align}
  \tilde{y}(n,\tau) &= \text{IDTFT}(Y(\omega, \tau))
  \\
  y(n) &= \sum_\tau f(n - \tau R)\tilde{y}(n - \tau R,\tau)
\end{align}

Note that synthesizing a single sample of $y(n)$ requires overlapping
$\frac{M}{R}$ frames which adds a latency of $M$ samples between $y$ and it's sources.

In the absence of spectral modifications, 
we can achieve perfect reconstruction 
of a signal transformed by the STFT and ISTFT given
that the synthesis and analysis windows obey
\begin{align}
  \sum_\tau w(n - \tau R)f(n - \tau R) = 1,\; \forall n\in \mathbb{Z}
\end{align}
I chose to use the square root of the Hann window as both the analysis and frequency window
which gives perfect reconstruction when $R = \frac{M}{2k}$ for $k\in\mathbb{Z}$.
\begin{align}
  w(n) = 
  f(n) =
  \cos(\pi n / M)
  %, 
  %n\in
  %\left[
  %-\frac{M-1}{2}, 
  %\frac{M-1}{2}
%\right]
\end{align}

For actual computation we replace, the DTFT and IDTFT with the FFT and IFFT respectively. 
The FFT has $M$ bins $n\in\left[-\frac{M - 1}{2}, \frac{M - 1}{2}\right]$.
The frequency of the $n$th bin is given by
\begin{align}
  \omega = \frac{f_s n}{M}
\end{align}
where $f_s$ is sampling rate of the signal.
Since the audio signal $x(n)$ is real, $X(\omega)$ is conjugate symmetric, so we only need to retain the positive bins $\left[0, \frac{M - 1}{2}\right]$ of the FFT.


\subsection{Spectrum Segmentation}

As we 
In order to resolve the vertical incoherence --- the phasing issues within the frame --- my solution like others is to lock regions of the spectrum where phase relations are important. Rather than performing a transport over all bins, bins which are used to represent the same spectral information are lumped together. 

TODO
Many applications use simple peak finding, which is good in most appliccations but fails in certain settings.
Instead I determine where to split the window based on the \emph{reassigned frequency}.
The reas


In order to fix the spec
Techniques mention peak finding.
However peak finding does not give a great sense of where to put boundaries. 
The reassigned frequency can be computed by:
\begin{align}
  \hat{\omega}(\omega) = 
  \omega + 
  \Im\left\{%
    \frac{X_\mathcal{D}(\omega)\cdot X^*(\omega)}{|X(\omega)|^2}
  \right\}
\end{align}
where $X_\mathcal{D}$ is the spectrum of the $x$ with the window
$w_\mathcal{D}(n) = \frac{d}{dt}w(n)$:
\begin{align}
  X_\mathcal{D}(\omega) = \fft\left(w_\mathcal{D}(n)x(n)\right)
\end{align}
The derivative of our chosen window, the square root of the Hann window is:
\begin{align}
   w_\mathcal{D}(n) = -\frac{\pi f_s}{M}\sin\left(\frac{\pi n}{M}\right)
\end{align}

TODO: Plot of segmentation
TODO: Formalize segmentation algorithm

This algorithm segments the spectrum into how at at most $N$ segments $s\in S$.
Each is has a center $c(s)$ and a set of indices $U(s)$.
The mass of each segment is given by 
\begin{align}
  \rho(s) = \frac{\sum_{n\in U(s)}|X(n)|}{\sum_{n}|X(n)|}
\end{align}
Note that
\begin{align}
  \sum_{s\in S} \rho(s) = 1
\end{align}

\subsection{Transport}

After two audio signals have been segmented into $S_0$ and $S_1$
we then compute a transformation matrix
$T\in \mathbb{R}^{S_0\times S_1}$
whose entries represents the transfer of mass from one segment to the other. 
We want to construct this matrix so that the total movement of mass (or work) is minimized.
%We want this matrix to minimize the distance the centers of each segment move.
We do not want to transport a negative mass or more mass than is available:
\begin{align}
  0 < T(s_0, s_1) \leq \max(\rho(s_0), \rho(s_1))
\end{align}
Additionally, we want to constrain that all of the mass receives an assignment:
\begin{align}
  \rho(s_0) = &\sum_{s_1\in S_1} T(s_0, s_1)
  \;
  \forall s_0\in S_0
  \\
  \rho(s_1) = &\sum_{s_0\in S_0} T(s_0, s_1)
  \;
  \forall s_1\in S_1
\end{align}
%We want to construct a set of assignments $T$, where each assignment $n\in T$ 
%assigns a mass $\rho_n$ to be transported between segments $m_n^0$ and $m_n^1$.
%\begin{align}
  %0 < \rho_n \leq \max(\rho_{m_n^0}, \rho_{m_n^1})
%\end{align}
%Additionally, we want to constrain that all of the mass receives an assignment:
%\begin{align}
  %\rho_m = &\sum_{\substack{n\in T,\\m_n^0 = m}} \rho_n 
  %\;
  %\forall m\in S^0
  %\\
  %\rho_m = &\sum_{\substack{n\in T,\\m_n^1 = m}} \rho_n 
  %\;
  %\forall m\in S^1
%\end{align}
Given these constraints we want to find the assignment $T$ that minimizes the 2-Wasserstein distance between the segments.
\begin{align}
  \min_T &\sum_{\substack{s_0\in S^0\\ s_1\in S^1}} T(s_0, s_1)|c(s_0) - c(s_1)|^2
\end{align}
We compute such a $T$ with algorithm~\ref{transport}.
%We can satisfy the constrains and approximate the using the following algorithm:

\begin{algorithm}
  \caption{Compute Transformation Matrix}\label{transport}
\begin{algorithmic}
  \State{$T \leftarrow 0$}
  \State{$i,j \leftarrow 0$}
  \State{$\rho_0 \leftarrow s_0^i$}
  \State{$\rho_1 \leftarrow s_1^j$}
  \While{$i < |S_0|$ and $j < |S_1|$}
    \If{$\rho_0 < \rho_1$}
      \State{$T(s_0^i, s_1^j) \leftarrow \rho_0$}
      \State{$i \leftarrow i + 1$}
      \State{$\rho_1 \leftarrow \rho_1 - \rho_0$}
      \State{$\rho_0 \leftarrow \rho(s_0^i)$}
    \Else
      \State{$T(s_0^i, s_1^j) \leftarrow \rho_1$}
      \State{$j \leftarrow j + 1$}
      \State{$\rho_0 \leftarrow \rho_0 - \rho_1$}
      \State{$\rho_1 \leftarrow \rho(s_1^i)$}
    \EndIf
  \EndWhile
\end{algorithmic}
\end{algorithm}

TODO: Plot of overlapping pieces

TODO: Proof - talk to Justin

%Using this algorithm, the number of assignments is $2N - 1$.

%From this point we could perform an interpolation.
%With interpolation factor $k$ the new peak index would be
%\begin{align}
  %c' = (1 - k)c(s_0) + k(c(s_1))
%\end{align}
%However we must first deal with the phase
\newpage
\subsection{Phase Accumulation}

When we move a segment of the spectrum to a new pitch, 
the phases within that section rotate either slower or faster.
This change does not make an audible difference within a window itself,
but it can cause interference in the overlap between windows known as \emph{horizontal incoherence}.

Include plot

One solution to this in time-stretching is to estimate the phase of the next frame based on its new instantaneous frequency. 
If a sinusoid with phase $\theta(\omega, \tau)$ 
is oscillating with instantaneous frequency $\hat{\omega}(\omega)$ over frame $\tau$, we would expect the phase of the next frame to be
\begin{align}
  \theta(\omega, \tau+1) = 
  \theta(\omega,\tau) + \frac{R}{f_s}\hat{\omega}(\omega)
\end{align}
This model is good for signals which do not undergo rapid changes in frequency. 
However it does a poor job of dealing with transients in audio that can happen from sharp note attacks and percussive sounds. 
To cope with this, people employ reinitialization steps in frames where transients are detected, however this method is.

Fortunately however, we have at our disposal the phases of not one but two pitches.
The center of  will oscillate at $(1 - k)\hat{\omega}(m_0) + \hat{\omega}^1_m$.
Therefore we can interpolate.

For every bin $i$ keep track of the accumulated phase $\theta_i^\tau$ which we want to have the following properties:

\begin{align}
  \Theta_i^\tau &\approx
  \sum_{t = 0}^\tau 
  \frac{R}{f_s}\hat{\omega}_i^t\\
  \Theta_i^\tau &= \theta_i^\tau\pmod{2\pi}
\end{align}



Therefore we want the synthesis phase of $\theta_i$ at the center of mass of segment $m$ is
\begin{align}
  \theta_{c_m} = (1-k)\Theta_{c_m^0} + k\Theta_{c_m^1}
\end{align}

To preserve the vertical coherence between all bins within a singular
we add $\theta_{c_m}' - \theta_{c_m}$ to all pitches
\begin{align}
  \theta_i' = \theta_i + \theta_{c_m}' - \theta_{c_m}
\end{align}

This way all the local relationships between phases are kept.

\subsection{Resynthesis}

Compute new center.
Compute new phase for horizontal coherence
Shift phases to preserve vertical coherence
linearly interpolate between the resulting

Once the phases of each segment have been centered around the synthesis phase,
we can synthesize a new spectrum:

\begin{algorithmic}
\\
\For{all $n$}
  \State{$Y(n) \leftarrow 0$}
  \Comment{Clear the output}
\EndFor
\\
\For{$m_0\in S_0$, $m_1\in S_1$}
  \\
  \State{$\hat{n} \leftarrow (1 - k)c(m_0) + kc(m_1)$}
  \Comment{$\hat{n}$ is the new COM index}
  \State{$\hat{\Theta} \leftarrow (1 - k)\Theta(c(m_0)) + k\Theta(c(m_1))$}
  \Comment{$\theta'$ is the new COM phase}
  \\
  \For{$n\in U(m_0)$}
    \\
    \State{$n' \leftarrow n + \hat{n} - c(m_0)$}
    \State{$\Theta' \leftarrow \Theta_0(n) + \hat{\Theta} - \Theta_0(c(m_0))$}
    \\
    \State{%
      $Y(n') 
      \leftarrow 
      Y(n') +
      (1 - k)
      \frac{T(m_0, m_1)}{\rho(m_0)}
      |X_0(n)|
      e^{i\theta'}$
    }
    \\
  \EndFor
\EndFor
\end{algorithmic}


reentered by the phase accumulation, 
we can simply linearly interpolate the audio and add it to the spectrum.
Since the 

\section{Implementation}

I implemented the above algorithm in real-time as an audio effect in C++.
The program listens to two streams of audio
which can be sent to it from a any diginal audio workstation, like Ableton Live.
It then produces a new stream of audio 
whose spectrum is transported between the two
according to a MIDI value.
The program depends on FFTW (Fastest Fourier Transform in the West) for Fourier transforms, and the open source libraries PortAudio and PortMidi for audio and MIDI handling respectively.

I work with a window size of 4096 samples and a sampling rate of 10 milliseconds. This is audible. 

My code is available at
\url{https://github.com/sportdeath/Vocoder}

A video of my implimentation is available at.

\section{Conclusion}

If one of the sources is time varying the source material quickly deteriorates.
Great for constant sounds.

\section{References}

Fundemental theory

that phase vocoder one

1d transport paper?

STFT

weighted overlap add

fftw

portaudio

portmidi

cite justin solomon

\end{document}
